%Jennifer Pan, August 2011

\documentclass[10pt,letter]{article}
	% basic article document class
	% use percent signs to make comments to yourself -- they will not show up.
\usepackage{enumerate}
\usepackage{amsmath}
\usepackage{amssymb}
	% packages that allow mathematical formatting

\usepackage{graphicx}
	% package that allows you to include graphics

\usepackage{setspace}
	% package that allows you to change spacing

\onehalfspacing
	% text become 1.5 spaced

\usepackage{fullpage}
	% package that specifies normal margins
	

\begin{document}
	% line of code telling latex that your document is beginning


\title{Homework 4}

\author{SABIC: Physics}

\date{Due February 29, 2016}
	% Note: when you omit this command, the current dateis automatically included
 
\maketitle 
	% tells latex to follow your header (e.g., title, author) commands.

\section*{Reading:}
Read Chapters 6 and 7.
\section*{Problem 1: practice with estimation}
\begin{enumerate}[(a)]
\item The gravitational potential of the sun on the Earth is given by $GM_{sun}M_{Earth}/x$, where $G =6.67 \times 10^{-11} {\rm N m}^2/{\rm kg}^2$ is Newton's gravitational constant and $x = 1.49 \times 10^11 m$ is the distance from the Earth to the Sun. (i) Estimate the gravitatational force the Earth exerts on the sun. (ii) Estimate the speed at which the Earth rotates about the sun, and find the period of rotation.
\item A bullet from a $.50$ caliber rifle (bullet weight of $50$ grams) can be used to hit a target over $2$ km away. Estimate the potential energy in the gunpowder. (Hint: find the velocity of the bullet as it leaves the gun, and use conservation of energy.)
\end{enumerate}
\section*{Problem 2: Kinetic Energy and Work }
\begin{enumerate}[(a)]
\item Book problem $6.21$
\item Book problem $6.25$
\end{enumerate}
\section*{Problem 3: Conservation of Energy}
\begin{enumerate}[(a)]
\item Book problem $7.37$. Hint: the friction force does some work $W = F_{\rm friction} \cdot \Delta x$, so $\Delta E = W$.
\item Book problem $7.42$. Hint: the total acceleration at the top of the loop is due to gravity, pointing downwards (at the minimum speed the normal force, which also points downwards, is zero). What must this acceleration be equal to for the motion to be in a circle? Reviewing the equations for circular motion may help.
\end{enumerate}

\end{document}
	% line of code telling latex that your document is ending. If you leave this out, you'll get an error
