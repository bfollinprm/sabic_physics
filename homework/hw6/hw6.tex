%Jennifer Pan, August 2011

\documentclass[10pt,letter]{article}
	% basic article document class
	% use percent signs to make comments to yourself -- they will not show up.
\usepackage{enumerate}
\usepackage{amsmath}
\usepackage{amssymb}
	% packages that allow mathematical formatting

\usepackage{graphicx}
	% package that allows you to include graphics

\usepackage{setspace}
	% package that allows you to change spacing

\onehalfspacing
	% text become 1.5 spaced

\usepackage{fullpage}
	% package that specifies normal margins
	

\begin{document}
	% line of code telling latex that your document is beginning


\title{Homework 6}

\author{SABIC: Physics}

\date{Due April 20, 2016}
	% Note: when you omit this command, the current dateis automatically included
 
\maketitle 
	% tells latex to follow your header (e.g., title, author) commands.

\section*{Reading:}
Read chapter 8.
\section*{Problem 1: estimation}
\begin{enumerate}[(a)]
\item Estimate the force exerted on a golf ball by a driver hitting the ball $250$m down the fairway. Assume the ball is in contact with the driver for $1$ms.
\end{enumerate}
\section*{Problem 2: Conceptual}
\begin{enumerate}[(a)]
\item What hurts more: being hit by a basketball you catch, or one that bounces of your hands? why?
\item Question $Q8.20$ in the book.
\end{enumerate}
\section*{Problem 3: Linear momentum}
\begin{enumerate}[(a)]
\item Problem $8.50$
\item Problem $8.69$
\end{enumerate}
\section*{Problem 4: Impulse}
\begin{enumerate}[(a)]
\item Problem $8.12$
\item Problem $8.96$
\end{enumerate}

\end{document}
	% line of code telling latex that your document is ending. If you leave this out, you'll get an error
