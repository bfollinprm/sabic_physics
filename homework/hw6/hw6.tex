%Jennifer Pan, August 2011

\documentclass[10pt,letter]{article}
	% basic article document class
	% use percent signs to make comments to yourself -- they will not show up.
\usepackage{enumerate}
\usepackage{amsmath}
\usepackage{amssymb}
	% packages that allow mathematical formatting

\usepackage{graphicx}
	% package that allows you to include graphics

\usepackage{setspace}
	% package that allows you to change spacing

\onehalfspacing
	% text become 1.5 spaced

\usepackage{fullpage}
	% package that specifies normal margins
	

\begin{document}
	% line of code telling latex that your document is beginning


\title{Homework 5}

\author{SABIC: Physics}

\date{Due April 14, 2016}
	% Note: when you omit this command, the current dateis automatically included
 
\maketitle 
	% tells latex to follow your header (e.g., title, author) commands.

\section*{Reading:}
Review chapters 1 and 4.
\section*{Problem 1: practice with errors}
\begin{enumerate}[(a)]
\item A race car travels around a circular track of radius $130 \pm 3$m. A radar gun measures its velocity at $98 \pm 5$m/s. (i) How long does it take for the car to travel around the track? (ii) What is the acceleration of the car? Give answers including standard error.
\item A group of $10$ physics students climb to the top of a $20$m tower and drop stones. Using a stopwatch, each student times how long it takes for their stones to fall. The results are shown in the table below:
\begin{tabular}{ |p{6cm}||p{6cm}|  }
 \hline
 \multicolumn{2}{|c|}{Stone Drop} \\
 \hline
 Student & Measured time (s) \\
 \hline
 Ali & 2.508 \\
 Ben & 2.619 \\
 Carlina & 2.590 \\
 Dan & 2.425 \\
 Emily & 2.426 \\
 Francis & 2.367 \\
 Gia & 2.463 \\
 Hassan & 2.426 \\
 Injit &  2.627 \\
 Julia & 2.388 \\
 \hline
\end{tabular}
\newline
(i) What is the average time? What is the standard deviation from the average?
(ii) The students use these numbers to calculate the strength of gravity. What do they get (include error). Is there systematic error? What might have caused it?
\end{enumerate}
\section*{Problem 2: Unit vectors}
\begin{enumerate}[(a)]
\item As was done in class for the unit vector $\hat{r}$, show that for a particle moving counterclockwise about a point with position $r(t)$, the change $\delta \hat{\theta}$ over a short time $\delta t$ is given by 
\begin{equation*}
\Delta \hat{\theta} = \Delta \theta \hat{r}.
\end{equation*}
\end{enumerate}
\section*{Problem 3: Kepler's laws}
\begin{enumerate}[(a)]
\item Haley's comet orbits the sun with a period of $75$ years. (i) What is the semi-major axis of the orbit? (ii) The eccentricity $\epsilon = f/a$ of Haley's orbit is $0.967$, where $f = \sqrt{a^2 - b^2}$ is the distance from the center of the orbit to the focus. What is the average speed of the orbit? (iii) At what distance is the comet from the sun when its instantaneous speed matches the average speed? (iv) The comet moves fastest when it's closest to the sun. Using Kepler's second law 
\begin{equation}
r^2 \frac{d\theta}{dt} = r\frac{d v_\perp}{dt} = \text{ constant},
\end{equation} 
what speed is the comet moving at that time?
\end{enumerate}

\end{document}
	% line of code telling latex that your document is ending. If you leave this out, you'll get an error
