%Jennifer Pan, August 2011

\documentclass[10pt,letter]{article}
	% basic article document class
	% use percent signs to make comments to yourself -- they will not show up.
\usepackage{enumerate}
\usepackage{amsmath}
\usepackage{amssymb}
	% packages that allow mathematical formatting

\usepackage{graphicx}
	% package that allows you to include graphics

\usepackage{setspace}
	% package that allows you to change spacing

\onehalfspacing
	% text become 1.5 spaced

\usepackage{fullpage}
	% package that specifies normal margins
	

\begin{document}
	% line of code telling latex that your document is beginning


\title{Homework 5}

\author{SABIC: Physics}

\date{Due April 14, 2016}
	% Note: when you omit this command, the current dateis automatically included
 
\maketitle 
	% tells latex to follow your header (e.g., title, author) commands.

\section*{Reading:}
Review chapters 1 and 4.
\section*{Problem 1: practice with errors}
\begin{enumerate}[(a)]
\item A race car travels around a circular track of radius $130 \pm 3$m. A radar gun measures its velocity at $98 \pm 5$m/s. (i) How long does it take for the car to travel around the track? 
\newline
The time is given by $\bar{t} = 130/98 = 1.33$s, with error $\Delta t = 1.33 \sqrt{(3/130)^2 + (5/98)^2} = 0.07$s.
\newline
 (ii) What is the acceleration of the car? 
 \newline
 The acceleration is given by $\bar{a} = v/2t = d/2t^2 = 130/2*(1.32)^2 = 37.3 {\rm m/s}^2$. The error is given by $\Delta a = a \sqrt{(\Delta v/v)^2 + (2\Delta t/2t)^2} = 0.07 {\rm m/s}^2$.
\item A group of $10$ physics students climb to the top of a $20$m tower and drop stones. Using a stopwatch, each student times how long it takes for their stones to fall. The results are shown in the table below:
\begin{tabular}{ |p{6cm}||p{6cm}|  }
 \hline
 \multicolumn{2}{|c|}{Stone Drop} \\
 \hline
 Student & Measured time (s) \\
 \hline
 Ali & 2.508 \\
 Ben & 2.619 \\
 Carlina & 2.590 \\
 Dan & 2.425 \\
 Emily & 2.426 \\
 Francis & 2.367 \\
 Gia & 2.463 \\
 Hassan & 2.426 \\
 Injit &  2.627 \\
 Julia & 2.388 \\
 \hline
\end{tabular}
\newline
(i) What is the average time? What is the standard deviation from the average?
\newline
The average time $\bar{t}$is given by 
\begin{equation}
\bar{t} = \frac{1}{N}\sum_i t_i = 2.48 {\rm s}.
\end{equation}
The Standard deviation is given by
\begin{equation}
\sqrt{\frac{1}{N-1} \sum_i (t_i - \bar{t})^2} = 0.09.
\end{equation}
The students thus report the time for the stone to fall as $t = (2.48 \pm 0.09)$s. 
\newline
(ii) The students use these numbers to calculate the strength of gravity. What do they get (include error). Is there systematic error? What might have caused it?
\newline
The strength of gravity is given 
\begin{equation}
g = \frac{2\delta x}{t^2}. 
\end{equation}
Everything but the time is known exactly. So we have an average constraint on $g$ of 
\begin{equation}
\bar{g} = \frac{2 (20 {\rm m})}{(2.48 {\rm s})^2} = 6.48 {\rm m/s}^2.
\end{equation}
Now we need to find the uncertainty. First find the total uncertainty in the numerator $(0 {\rm m})$ and denominator $\sqrt{2 \Delta t ^2} = 0.13 {\rm s}$. Now, using the rule for combining errors in a ratio:
\begin{equation}
\Delta g = \bar{g} \sqrt{\left(\frac{0}{40}\right)^2 + \left(\frac{0.13}{2.48^2} \right)^2} = 0.94 {\rm m/s}^2.
\end{equation}
The students would report their measurment as $(6.48 \pm 0.94 ){\rm m/s}^2$. Since we know the acceleration due to gravity is really $9.8 {\rm m/s}^2$, the students measured a systematically lower acceleration than expected, due to a systematic error. Sources of error could be measuring times that were too long (by not properly using the stopwatch) or incorrectly measuring the height fallen (with a bad ruler, or forgetting to include the height of the person dropping the stone).
\end{enumerate}
\section*{Problem 2: Unit vectors}
\begin{enumerate}[(a)]
\item As was done in class for the unit vector $\hat{r}$, show that for a particle moving counterclockwise about a point with position $r(t)$, the change $\delta \hat{\theta}$ over a short time $\delta t$ is given by 
\begin{equation*}
\Delta \hat{\theta} = \Delta \theta \hat{r}.
\end{equation*}
\newline
We'll do this one in class.
\end{enumerate}
\section*{Problem 3: Kepler's laws}
\begin{enumerate}[(a)]
\item Haley's comet orbits the sun with a period of $75$ years. (i) What is the semi-major axis of the orbit? 
\newline
We can use Kepler's $3^rd$ law: 
\begin{equation}
T^2 = \frac{4 \pi^2}{G M_{\rm sun}}a^3.
\end{equation}
Plugging in the values of the constants and solving for $a$, we get $a = 2.66\times 10^{12}$m.
\newline
(ii) The eccentricity $\epsilon = f/a$ of Haley's orbit is $0.967$, where $f = \sqrt{a^2 - b^2}$ is the distance from the center of the orbit to the focus. What is the average speed of the orbit? 
\newline
We can find $b$ from the eccentricity by re-arranging the relation
\begin{equation}
b = a \sqrt{ 1 - \epsilon^2} = 4.83 \times 10 ^ {11} {\rm m}.
\end{equation}
Using an online calculator to find the circumference of the resulting ellipse, we get $C = 1.11 \times 10^13${\rm m}. That gives us an average velocity of $\bar{v} = 4690$m/s.
\newline
(iii) At what distance is the comet from the sun when its instantaneous speed matches the average speed? 
\newline
We need $|\vec{v}(t)| = 4690$m/s. In general,
\begin{equation}
\vec{v}(t) = \frac{dr}{dt} \hat{r} \pm r \frac{d\theta}{dt}\hat{\theta}
\end{equation}
We know $r(t) = \frac{p}{1 + \epsilon \cos{\theta}}$, so 
\begin{equation}
\frac{dr}{dt} = \frac{p \epsilon \sin \theta}{(1 + \epsilon \cos \theta)^2}\frac{d\theta}{d t}.
\end{equation}
We have $ p = b^2/a = 8.78 \times 10^{10}$m, and $\epsilon = 0.973$. Plugging everything in, we get 
\begin{equation}
\vec{v}(t) = \left(\frac{8.50 \times 10^{10} \sin \theta}{(1 + 0.973 \cos \theta)^2} \hat{r} \pm r \hat{\theta}\right)\frac{d\theta}{dt}
\end{equation}
or 
\begin{equation}
v(t) = \frac{d \theta}{dt}\sqrt{\left(\frac{8.50 \times 10^{10} \sin \theta}{(1 + 0.973 \cos \theta)^2}\right)^2 + r^2}
\end{equation}
\newline
(iv) The comet moves fastest when it's closest to the sun. Using Kepler's second law 
\begin{equation}
r^2 \frac{d\theta}{dt} = rv_\perp = \text{ constant},
\end{equation} 
what speed is the comet moving at that time?
\newline
The constant is such that over a full period $\Delta t = T$, $\Delta \theta = 2 \pi$ The average radius is $r^2 = ab$, so the constant is $nab = \frac{2\pi}{T}ab$.
We have $r = r_min = a - f = 4.43\times10^{10}$m. The velocity is solely in the $\theta$ direction here, so $r v = \frac{2\pi}{T}ab$. This implies that
$v = \frac{2 \pi a b}{rT} = 77,100$m/s.

\end{enumerate}

\end{document}
	% line of code telling latex that your document is ending. If you leave this out, you'll get an error
