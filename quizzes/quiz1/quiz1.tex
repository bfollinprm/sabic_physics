\documentclass[12pt]{article}

\setlength{\topmargin}{-.75in} \addtolength{\textheight}{2.00in}
\setlength{\oddsidemargin}{.00in} \addtolength{\textwidth}{.75in}

\usepackage{amsmath,color,graphicx}

\nofiles

\pagestyle{empty}

\setlength{\parindent}{0in}


\begin{document}

\noindent {\sc {\bf {\Large Quiz 1}}
            \hfill SABIC Physics, Winter 2016}
\bigskip

\noindent {\sc  {}
            \hfill {\large Name:}
             \hfill}
\bigskip

{\bf Problem 1.}(12 points.) Short answer--no more than one sentence each. 
\begin{enumerate}
\item Under what conditions is instantaneous velocity equal to average velocity?
\bigskip
\bigskip
\bigskip
\bigskip
\bigskip
\item In uniform circular motion, what are the average velocities and acceleration after one full rotation?
\bigskip
\bigskip
\bigskip
\bigskip
\bigskip
\item An elevator travels with speed $9.8$m/s upwards, and you drop a ball. What is the acceleration of the ball?
\bigskip
\bigskip
\bigskip
\bigskip
\bigskip
\end{enumerate}
{\bf Problem 2.}(14 points.) Motion in 1 dimension:

An electron leaves one end of a TV picture tube with zero initial speed and travels in a straight line to the accelerating grid, which is $1.80$ cm away. It reaches the grid with a speed of $3 \times 10^6$m/s. If the accelerating force is constant, compute (a) the acceleration; (b) the time to reach the grid; (c) the net force, in newtons (the mass of an electron is $9 \times 10^{-31}$kg). (You can ignore the gravitational force on the electron.)

\bigskip
\bigskip
\bigskip
\bigskip
\bigskip
\bigskip
\bigskip
\bigskip
\bigskip
\bigskip
\bigskip
\bigskip
\bigskip
\bigskip
\bigskip
\bigskip
\bigskip
\bigskip
\bigskip
\bigskip
\bigskip
\bigskip
\bigskip
\bigskip
\newpage
{\bf Problem 3.}(14 points.) Forces and acceleration:
\begin{enumerate}
\item Two horses pull horizontally on ropes attached to a stump. The two forces $\vec{\bf F}_1$ and $\vec{\bf F}_2$ result in a total force $\vec{\bf R}$, with magnitude half of $\vec{\bf F}_1$. Let $F_1 = 1200$N and let $\vec{\bf R}$ make an angle of $60^\circ$ with $\vec{\bf F}_1$. (i) What is the magnitude and direction of $\vec{\bf F}_2$? If you pick a coordinate system, you may give your answer in terms of $F_y$, $F_x$. 
\bigskip
\bigskip
\bigskip
\bigskip
\bigskip
\bigskip
\bigskip
\bigskip
\bigskip
\bigskip
\bigskip
\bigskip
\bigskip
\bigskip
\bigskip
\item If the stump weighs $200$N under gravity ($g = 10{\rm m/s}^2$, and there is a $200$N friction force opposing the motion of the stump, what is the acceleration of the stump?
\bigskip
\bigskip
\bigskip
\bigskip
\bigskip
\bigskip
\bigskip
\bigskip
\bigskip
\bigskip
\bigskip
\end{enumerate}
\end{document}
