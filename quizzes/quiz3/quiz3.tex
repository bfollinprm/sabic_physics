\documentclass[12pt]{article}

\setlength{\topmargin}{-.75in} \addtolength{\textheight}{2.00in}
\setlength{\oddsidemargin}{.00in} \addtolength{\textwidth}{.75in}

\usepackage{amsmath,color,graphicx}
\usepackage{enumerate}
\usepackage{multicol}
\nofiles

\pagestyle{empty}

\setlength{\parindent}{0in}


\begin{document}

\noindent {\sc {\bf {\Large Quiz 1}}
            \hfill SABIC Physics, Spring 2016}
\bigskip

\noindent {\sc  {}
            \hfill {\large Name:}
             \hfill}
\bigskip

{\bf Problem 1.}(12 points.) Short answer--no more than one sentence each. 
\begin{enumerate}[(a)]
\item An asteroid has an orbital period of $4.62$ years and an eccentricity of $0.233$. Find the semi-major axis of the orbit.
\bigskip
\bigskip
\bigskip
\bigskip
\bigskip
\item The Earth exerts a force on the Sun exactly as the Sun exerts a force on the Earth. Why is it that we say the Earth orbits the sun and not the other way around?
\bigskip
\bigskip
\bigskip
\bigskip
\bigskip
\item The sun pulls on the moon with twice the gravitational force as the Earth pulls on the moon. Why, then, doesn't the sun take the moon away from the earth?
\bigskip
\bigskip
\bigskip
\bigskip
\bigskip
\end{enumerate}
{\bf Problem 2.}(12 points.) 
A planet was discovered ordbiting around another star. Its maximum orbital distance was measured to be $12$ million kilometers from the center of the star, and its orbital period was extimated at $6.3$ days. The eccentricity of the planet's orbit was given by $\epsilon = 0.36$
\begin{enumerate}[(a)]
\item Find $a$, $b$, and write the equation describing the orbit of the planet.
\bigskip
\bigskip
\bigskip
\bigskip
\bigskip
\bigskip
\bigskip
\bigskip
\bigskip
\bigskip
\bigskip
\bigskip
\item What is the mass of the star HD $68988$? Express your answer in terms of the mass of the sun.
\bigskip
\bigskip
\bigskip
\bigskip
\bigskip
\bigskip
\bigskip
\end{enumerate}
{\bf Problem 3.}(16 points) 
The international space station (ISS) orbits the earth with an orbital period of $90$ minutes.  
\begin{enumerate}[(a)]
\item What is the semimajor axis of the orbit? 
\bigskip
\bigskip
\bigskip
\bigskip
\bigskip
\item Assuming the motion is circular, so that $\frac{dr}{dt} = 0$, what speed is the ISS traveling?
\bigskip
\bigskip
\bigskip
\bigskip
\bigskip
\bigskip
\bigskip
\bigskip
\item What acceleration does the ISS feel due to the Earth's gravitational pull? Is it a good approximation to call an astronaut walking outside the ISS weightless? Why or why not?
\bigskip
\bigskip
\bigskip
\bigskip
\bigskip
\bigskip
\bigskip
\bigskip
\bigskip
\bigskip
\item In the interest of an experiment on changing gravitational fields, the world decides to send the ISS on an elliptical orbit with ellipticity $\epsilon = 0.64$, while keeping the radius of nearest approach to the Earth at the same distance as before. What is the new period of the orbit?
\bigskip
\bigskip
\bigskip
\bigskip
\bigskip
\bigskip
\bigskip
\end{enumerate}
\section*{Constants}
\begin{multicols}{2}
\begin{itemize}
\item $G = 6.7 \times 10^{-11} \frac{{\rm N m}^2}{{\rm kg}^2}$
\item $M_{\rm sun} = 2 \times 10^{30}$kg 
\item distance from sun to earth (semi-major axis): $1.5 \times 10^{9}$m
\item Earth-orbit eccentricity: $0.017$
\item $M_{\rm earth} = 6.0 \times 10^{24}$kg
\item radius of the Earth: $6.4 \times 10^6$m
\item $\kappa = \frac{4\pi^2}{GM}$ 
\end{itemize}
\end{multicols}
\end{document}
